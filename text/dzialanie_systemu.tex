% encoding: utf8
% !TEX encoding = utf8
% !TeX spellcheck = pl_PL

\chapter{Eksperymenty\label{chap:weryfikacja_systemu}}
\graphicspath{{./velma/przerobione_testy/out/}{./images}}



W celu zaprezentowania działania nowego algorytmu kompensacji grawitacji uruchomiono program testowy w~trybie symulacji. Wszystkie eksperymenty wykazywały, że przy odpowiednio długim czasie eksperymentu uchyb końcówki dąży do zera. Do obliczeń przyjęto krótsze obserwacje przebiegów, w~celu pokazania zachowania systemu w~warunkach które wymagają stosunkowo szybkiej manipulacji.  Eksperymenty opisane są w~bazowym układzie odniesienia (rys. \ref{fig:velma2}).  Ocena jakości sterowania następuje poprzez porównanie przebiegów trajektorii realizowanych w~opisanych wcześniej wariantach.

W pracy zamieszczono wyniki eksperymentów o~tych samych trajektoriach zadanych w~czterech wariantach:
\begin{itemize}
	\item Bez algorytmu kompensacji, bez przedmiotu.
	\item Z~algorytmem kompensacji, bez przedmiotu.
	\item Z~algorytmem kompensacji, z~chwyconą puszką o~kształcie walca, masie 1 kg i~jednolitym rozkładzie mas.
	\item Z~algorytmem kompensacji, z~chwyconą wiertarką o~nierównomiernym kształcie i~masie 1,5 kg.
\end{itemize}
Eksperymenty wykonywane bez przedmiotu i bez załączonego algorytmu kompensacji mają na celu pokazanie trajektorii która jest wykonywana przez niezmodyfikowane prawo sterowania. Pozwala to na określenie pogorszenia jakości regulacji, gdy algorytm kompensacji grawitacji zostanie załączony. 

W~pracy nie zamieszczono wyników działania algorytmu sterowania bez kompensacji i~z chwyconymi przedmiotami. Chwytak robota zatrzymywał się na stałe w~pozycji chwytu przedmiotu, a~takie trajektorie nie były  w~żadnym stopniu wykonywane. 

\section{Dobór parametrów regulatora}
W trakcie wszystkich eksperymentów wszystkie parametry algorytmu sterowania były takie same. Macierze sprężystości i~sztywności miały takie same niskie wartości. Dobór parametrów członu całkującego nastąpił eksperymentalnie przy założeniu, że wszystkie parametry macierzy diagonalnej członu będą miały te same wartości. W~trakcie doboru wartości parametru członu całkującego przeprowadzono eksperyment podnoszenia przedmiotu (rys. \ref{fig:param_a}, \ref{fig:param_rot}). Poszukiwana wartość parametru nie powinna spowodować by człon całkujący zdominował inne człony oraz aby nie wprowadził dużych oscylacji. Nie może mieć za dużej wartości, gdyż wtedy ramiona robota usztywnią się w~trakcie manipulacji. Dobór parametów wykonywany jest w trakcie podnoszenia puszki. Parametr członu całkującego musi być na tyle dużej wartości by następowała kompensacja grawitacji testowego przedmiotu o~masie 1 kg w~rozsądnym czasie. 

Dla parametru o~wartości 50 widać wyraźne oscylacje widoczne zarówno w~położeniu jak i~w obrocie końcówki. Dla parametru o~wartości 5 regulacja następuje zbyt wolno. Nie pozwoli to na sprawne manipulowanie narzędziem jeśli jego parametry będą się zmieniać często. Dla wyłączonej kompensacji grawitacji (parametr równy 0) przedmiot nie jest podnoszony. Finalnie wybrano wartość parametru równą 10 jako pierwszą, która gwarantuje kompensację siły grawitacji w~rozsądnym czasie. 

\begin{figure}[H]
	\centering
	\subfigure[Oś $X$]{
		\label{fig:param_ax}
		\includegraphics[width=.47\linewidth]{./velma/przerobione_testy/out/param/common_ax.png}
	}
	\hfill
	\subfigure[Oś $Y$]{
		\label{fig:param_ay}
		\includegraphics[width=.47\textwidth]{./velma/przerobione_testy/out/param/common_ay.png}
	}
	\subfigure[Oś $Z$]{
		\label{fig:param_az}
		\includegraphics[width=.47\textwidth]{./velma/przerobione_testy/out/param/common_az.png}
	}

	\caption{Podnoszenie przedmiotu. Porównanie trajektorii z~różnymi parametrami członu całkującego.}
	\label{fig:param_a}
	
\end{figure}


\begin{figure}[H]
	\centering
	\subfigure[Kąt obrotu osi $X$]{
		\label{fig:param_rotx}
		\includegraphics[width=.47\textwidth]{./velma/przerobione_testy/out/param/common_rotx.png}
	}
	\hfill
	\subfigure[Kąt obrotu osi $Y$]{
		\label{fig:param_roty}
		\includegraphics[width=.47\textwidth]{./velma/przerobione_testy/out/param/common_roty.png}
	}
	
	
	\subfigure[Kąt obrotu osi $Z$]{
		\label{fig:param_rotz}
		\includegraphics[width=.47\textwidth]{./velma/przerobione_testy/out/param/common_rotz.png}
	}
	
	\caption{Podnoszenie przedmiotu. Porównanie parametrów członu całkującego jako kątów w~notacji Eulera w~zależności od czasu.}
	\label{fig:param_rot}
	
\end{figure}



\section{Podnoszenie przedmiotu}

Eksperyment ma przetestować zachowanie algorytmu kompensacji przy podnoszeniu przedmiotu o~nieznanych parametrach ruchach (rys. \ref{fig:podn_a}, \ref{fig:podn_rot}). Ruch widoczny jest głównie w~osi $Z$.

Trajektoria ruchu w~rzucie na wprost ruchu została zaprezentowana na rys. \ref{fig:podn_porow_komp} i~\ref{fig:podn_porow_przedm}. Trajektoria widoczna z~boku (w osiach $X$ oraz $Z$) została zaprezentowana na rys. \ref{fig:podn_porow_komp_bok},i \ref{fig:podn_porow_przedm_bok}.


\begin{figure}[H]
	\centering
	\subfigure[Oś $X$]{
		\label{fig:podn_ax}
		\includegraphics[width=.47\textwidth]{./velma/przerobione_testy/out/podn/common_ax.png}
	}
	\hfill
	\subfigure[Oś $Y$]{
		\label{fig:podn_ay}
		\includegraphics[width=.47\textwidth]{./velma/przerobione_testy/out/podn/common_ay.png}
	}
	
	\subfigure[Oś $Z$]{
		\label{fig:podn_az}
		\includegraphics[width=.47\textwidth]{./velma/przerobione_testy/out/podn/common_az.png}
	}

	\caption{Podnoszenie przedmiotu. Porównanie trajektorii pozycji w~zależności od czasu.}
	\label{fig:podn_a}

\end{figure}

\begin{figure}[H]
	\centering
	\subfigure[Brak algorytmu kompensacji]{
		\includegraphics[width=.47\textwidth]{./velma/przerobione_testy/out/podn/yz_ate_plot_podnoszenie_miekki_bez_brak.png}
	}
	\hfill
	\subfigure[Załączony algorytm kompensacji]{
		\includegraphics[width=.47\textwidth]{./velma/przerobione_testy/out/podn/yz_ate_plot_podnoszenie_miekki_komp_brak.png}
	}
	\caption{Podnoszenie przedmiotu. Porównanie trajektorii chwytaka w~osiach $Y$ i~$Z$}
	\label{fig:podn_porow_komp}
\end{figure}

\begin{figure}[H]
	\centering
	\subfigure[Kąt obrotu osi $X$]{
		\label{fig:podn_rotx}
		\includegraphics[width=.47\textwidth]{./velma/przerobione_testy/out/podn/common_rotx.png}
	}
	\hfill
	\subfigure[Kąt obrotu osi $Y$]{
		\label{fig:podn_roty}
		\includegraphics[width=.47\textwidth]{./velma/przerobione_testy/out/podn/common_roty.png}
	}
	

	\subfigure[Kąt obrotu osi $Z$]{
		\label{fig:podn_rotz}
		\includegraphics[width=.47\textwidth]{./velma/przerobione_testy/out/podn/common_rotz.png}
	}

	\caption{Podnoszenie przedmiotu. Porównanie trajektorii kątów w~notacji Eulera w~zależności od czasu.}
	\label{fig:podn_rot}

\end{figure}



\begin{figure}
	\centering
	\subfigure[Trajektoria z~chwyconą puszką]{
		\includegraphics[width=.47\textwidth]{./velma/przerobione_testy/out/podn/yz_ate_plot_podnoszenie_miekki_komp_piwo.png}
	}
	\hfill
	\subfigure[Trajektoria z~chwyconą wiertarką]{
		\includegraphics[width=.47\textwidth]{./velma/przerobione_testy/out/podn/yz_ate_plot_podnoszenie_miekki_komp_wiertarka.png}
	}
	\caption{Podnoszenie przedmiotu. Porównanie trajektorii chwytaka w~osiach $Y$ i~$Z$}
	\label{fig:podn_porow_przedm}
\end{figure}


\begin{figure}[H]
	\centering
	\subfigure[Brak algorytmu kompensacji]{
		\includegraphics[width=.47\textwidth]{./velma/przerobione_testy/out/podn/xz_ate_plot_podnoszenie_miekki_bez_brak.png}
	}
	\hfill
	\subfigure[Załączony algorytm kompensacji]{
		\includegraphics[width=.47\textwidth]{./velma/przerobione_testy/out/podn/xz_ate_plot_podnoszenie_miekki_komp_brak.png}
	}
	\caption{Podnoszenie przedmiotu. Porównanie trajektorii chwytaka w~osiach $X$ i~$Z$}
	\label{fig:podn_porow_komp_bok}
\end{figure}

\begin{figure}[H]
	\centering
	\subfigure[Trajektoria z~chwyconą puszką]{
		\includegraphics[width=.47\textwidth]{./velma/przerobione_testy/out/podn/xz_ate_plot_podnoszenie_miekki_komp_piwo.png}
	}
	\hfill
	\subfigure[Trajektoria z~chwyconą wiertarką]{
		\includegraphics[width=.47\textwidth]{./velma/przerobione_testy/out/podn/xz_ate_plot_podnoszenie_miekki_komp_wiertarka.png}
	}
	\caption{Podnoszenie przedmiotu. Porównanie trajektorii chwytaka w~osiach $X$ i~$Z$}
	\label{fig:podn_porow_przedm_bok}
\end{figure}

% \begin{figure}[H]
% 	\centering
% 	\subfigure[Rzut na wprost]{
% 		\label{fig:podn_porow_zbiorcze_a}
% 		\includegraphics[width=.47\textwidth]{./velma/przerobione_testy/out/podn/common_yz.png}
% 	}
% 	\hfill
% 	\subfigure[Rzut z~boku]{
% 		\label{fig:podn_porow_zbiorcze_b}
% 		\includegraphics[width=.47\textwidth]{./velma/przerobione_testy/out/podn/common_xz.png}
% 	}
% 	\caption{Porównanie wszystkich trajektorii bez zaznaczonego błędu}
% 	\label{fig:podn_porow_zbiorcze}
% \end{figure}




\section{Ruch ósemkowy}
Eksperyment ma przetestować zachowanie algorytmu kompensacji przy skomplikowanych ruchach (rys. \ref{fig:osemka_a}, \ref{fig:osemka_rot}).  Ruch ósemkowy zadany jest w~osi $Y$ oraz $Z$.
Trajektoria jest zadana zgodnie ze wzorem lemniskaty Bernoullego:
\begin{equation}
(y^2 + z^2)^2 = 2a^2(y^2-z^2)
\end{equation}

gdzie:
\begin{itemize}
	\item $y$ oraz $z$ to współrzędne trajektorii
	\item $a$ to parametr równania
\end{itemize}

Trajektoria ruchu w~rzucie ATE ruchu została zaprezentowana na rys. \ref{fig:osemka_porow_komp} i~\ref{fig:osemka_porow_przedm}. Trajektoria widoczna z~boku (w osiach $X$ oraz $Z$) została zaprezentowana na rys. \ref{fig:osemka_porow_komp_bok}, \ref{fig:osemka_porow_przedm_bok}.
\begin{figure}[H]
	\centering
	\subfigure[Oś $X$]{
		\label{fig:osemka_ax}
		\includegraphics[width=.47\textwidth]{./velma/przerobione_testy/out/osemka/common_ax.png}
	}
	\hfill
	\subfigure[Oś $Y$]{
		\label{fig:osemka_ay}
		\includegraphics[width=.47\textwidth]{./velma/przerobione_testy/out/osemka/common_ay.png}
	}
	
	
	\subfigure[Oś $Z$]{
		\label{fig:osemka_az}
		\includegraphics[width=.47\textwidth]{./velma/przerobione_testy/out/osemka/common_az.png}
	}

	\caption{Ruch ósemkowy. Porównanie trajektorii pozycji w~zależności od czasu.}
	\label{fig:osemka_a}

\end{figure}

\begin{figure}[H]
	\centering
	\subfigure[Brak algorytmu kompensacji]{
		\includegraphics[width=.47\textwidth]{./velma/przerobione_testy/out/osemka/yz_ate_plot_podnoszenie_miekki_bez_brak.png}
	}
	\hfill
	\subfigure[Załączony algorytm kompensacji]{
		\includegraphics[width=.47\textwidth]{./velma/przerobione_testy/out/osemka/yz_ate_plot_podnoszenie_miekki_komp_brak.png}
	}
	\caption{Porównanie trajektorii chwytaka w~osiach $Y$ i~$Z$}
	\label{fig:osemka_porow_komp}
\end{figure}


\begin{figure}[H]
	\centering
	\subfigure[Kąt obrotu osi $X$]{
		\label{fig:osemka_rotx}
		\includegraphics[width=.47\textwidth]{./velma/przerobione_testy/out/osemka/common_rotx.png}
	}
	\hfill
	\subfigure[Kąt obrotu osi $Y$]{
		\label{fig:osemka_roty}
		\includegraphics[width=.47\textwidth]{./velma/przerobione_testy/out/osemka/common_roty.png}
	}
	
	
	\subfigure[Kąt obrotu osi $Z$]{
		\label{fig:osemka_rotz}
		\includegraphics[width=.47\textwidth]{./velma/przerobione_testy/out/osemka/common_rotz.png}
	}

	\caption{Ruch ósemkowy. Porównanie trajektorii kątów w~notacji Eulera w~zależności od czasu.}
	\label{fig:osemka_rot}

\end{figure}


\begin{figure}[H]
	\centering
	\subfigure[Trajektoria z~chwyconą puszką]{
		\includegraphics[width=.47\textwidth]{./velma/przerobione_testy/out/osemka/yz_ate_plot_podnoszenie_miekki_komp_piwo.png}
	}
	\hfill
	\subfigure[Trajektoria z~chwyconą wiertarką]{
		\includegraphics[width=.47\textwidth]{./velma/przerobione_testy/out/osemka/yz_ate_plot_podnoszenie_miekki_komp_wiertarka.png}
	}
	\caption{Ruch ósemkowy. Porównanie trajektorii chwytaka w~osiach $Y$ i~$Z$}
	\label{fig:osemka_porow_przedm}
\end{figure}


\begin{figure}[H]
	\centering
	\subfigure[Brak algorytmu kompensacji]{
		\includegraphics[width=.47\textwidth]{./velma/przerobione_testy/out/osemka/xz_ate_plot_podnoszenie_miekki_bez_brak.png}
	}
	\hfill
	\subfigure[Załączony algorytm kompensacji]{
		\includegraphics[width=.47\textwidth]{./velma/przerobione_testy/out/osemka/xz_ate_plot_podnoszenie_miekki_komp_brak.png}
	}
	\caption{Ruch ósemkowy. Porównanie trajektorii chwytaka w~osiach $X$ i~$Z$}
	\label{fig:osemka_porow_komp_bok}
\end{figure}

\begin{figure}[H]
	\centering
	\subfigure[Trajektoria z~chwyconą puszką]{
		\includegraphics[width=.47\textwidth]{./velma/przerobione_testy/out/osemka/xz_ate_plot_podnoszenie_miekki_komp_piwo.png}
	}
	\hfill
	\subfigure[Trajektoria z~chwyconą wiertarką]{
		\includegraphics[width=.47\textwidth]{./velma/przerobione_testy/out/osemka/xz_ate_plot_podnoszenie_miekki_komp_wiertarka.png}
	}
	\caption{Ruch ósemkowy. Porównanie trajektorii chwytaka w~osiach $X$ i~$Z$}
	\label{fig:osemka_porow_przedm_bok}
\end{figure}

% \begin{figure}[H]
% 	\centering
% 	\subfigure[Rzut na wprost]{
% 		\label{fig:osemka_porow_zbiorcze_a}
% 		\includegraphics[width=.47\textwidth]{./velma/przerobione_testy/out/osemka/common_yz.png}
% 	}
% 	\hfill
% 	\subfigure[Rzut z~boku]{
% 		\label{fig:osemka_porow_zbiorcze_b}
% 		\includegraphics[width=.47\textwidth]{./velma/przerobione_testy/out/osemka/common_xz.png}
% 	}
% 	\caption{Ruch ósemkowy. Porównanie wszystkich trajektorii bez zaznaczonego błędu}
% 	\label{fig:osemka_porow_zbiorcze}
% \end{figure}

\section{Ruch w~bok}

Eksperyment ma przetestować zachowanie algorytmu kompensacji przy ruchu końcówki w~bok. Trajektoria ruchu w~rzucie ATE została zaprezentowana na rys. \ref{fig:w_bok_miekki_porow_komp} i~\ref{fig:w_bok_miekki_porow_przedm}.
% Trajektoria widoczna z~boku (w osiach $X$ oraz $Z$) została zaprezentowana na rys. \ref{fig:w_bok_miekki_porow_komp_bok}, \ref{fig:w_bok_miekki_porow_przedm_bok} i~\ref{fig:w_bok_miekki_porow_zbiorcze_b}.
\begin{figure}[H]
	\centering
	\subfigure[Oś $X$]{
		\label{fig:w_bok_miekki_ax}
		\includegraphics[width=.47\textwidth]{./velma/przerobione_testy/out/w_bok_miekki/common_ax.png}
	}
	\hfill
	\subfigure[Oś $Y$]{
		\label{fig:w_bok_miekki_ay}
		\includegraphics[width=.47\textwidth]{./velma/przerobione_testy/out/w_bok_miekki/common_ay.png}
	}
	
	\subfigure[Oś $Z$]{
		\label{fig:w_bok_miekki_az}
		\includegraphics[width=.47\textwidth]{./velma/przerobione_testy/out/w_bok_miekki/common_az.png}
	}

	\caption{Ruch w~bok. Porównanie trajektorii pozycji w~zależności od czasu.}
	\label{fig:w_bok_miekki_a}

\end{figure}

\begin{figure}[H]
	\centering
	\subfigure[Brak algorytmu kompensacji]{
		\includegraphics[width=.47\textwidth]{./velma/przerobione_testy/out/w_bok_miekki/yz_ate_plot_podnoszenie_miekki_bez_brak.png}
	}
	\hfill
	\subfigure[Załączony algorytm kompensacji]{
		\includegraphics[width=.47\textwidth]{./velma/przerobione_testy/out/w_bok_miekki/yz_ate_plot_podnoszenie_miekki_komp_brak.png}
	}
	\caption{Ruch w~bok. Porównanie trajektorii chwytaka w~osiach $Y$ i~$Z$.}
	\label{fig:w_bok_miekki_porow_komp}
\end{figure}


\begin{figure}[H]
	\centering
	\subfigure[Kąt obrotu osi $X$]{
		\label{fig:w_bok_miekki_rotx}
		\includegraphics[width=.47\textwidth]{./velma/przerobione_testy/out/w_bok_miekki/common_rotx.png}
	}
	\hfill
	\subfigure[Kąt obrotu osi $Y$]{
		\label{fig:w_bok_miekki_roty}
		\includegraphics[width=.47\textwidth]{./velma/przerobione_testy/out/w_bok_miekki/common_roty.png}
	}
	
	\subfigure[Kąt obrotu osi $Z$]{
		\label{fig:w_bok_miekki_rotz}
		\includegraphics[width=.47\textwidth]{./velma/przerobione_testy/out/w_bok_miekki/common_rotz.png}
	}

	\caption{Ruch w~bok. Porównanie trajektorii kątów w~notacji Eulera w~zależności od czasu.}
	\label{fig:w_bok_miekki_rot}

\end{figure}



\begin{figure}[H]
	\centering
	\subfigure[Trajektoria z~chwyconą puszką]{
		\includegraphics[width=.47\textwidth]{./velma/przerobione_testy/out/w_bok_miekki/yz_ate_plot_podnoszenie_miekki_komp_piwo.png}
	}
	\hfill
	\subfigure[Trajektoria z~chwyconą wiertarką]{
		\includegraphics[width=.47\textwidth]{./velma/przerobione_testy/out/w_bok_miekki/yz_ate_plot_podnoszenie_miekki_komp_wiertarka.png}
	}
	\caption{Ruch w~bok. Porównanie trajektorii chwytaka w~osiach $Y$ i~$Z$}
	\label{fig:w_bok_miekki_porow_przedm}
\end{figure}


% \begin{figure}
% 	\centering
% 	\subfigure[Brak algorytmu kompensacji]{
% 		\includegraphics[width=.47\textwidth]{./velma/przerobione_testy/out/w_bok_miekki/xz_ate_plot_podnoszenie_miekki_bez_brak.png}
% 	}
% 	\hfill
% 	\subfigure[Załączony algorytm kompensacji]{
% 		\includegraphics[width=.47\textwidth]{./velma/przerobione_testy/out/w_bok_miekki/xz_ate_plot_podnoszenie_miekki_komp_brak.png}
% 	}
% 	\caption{Porównanie trajektorii chwytaka w~osiach $X$ i~$Z$}
% 	\label{fig:w_bok_miekki_porow_komp_bok}
% \end{figure}

% \begin{figure}
% 	\centering
% 	\subfigure[Trajektoria z~chwyconą puszką]{
% 		\includegraphics[width=.47\textwidth]{./velma/przerobione_testy/out/w_bok_miekki/xz_ate_plot_podnoszenie_miekki_komp_piwo.png}
% 	}
% 	\hfill
% 	\subfigure[Trajektoria z~chwyconą wiertarką]{
% 		\includegraphics[width=.47\textwidth]{./velma/przerobione_testy/out/w_bok_miekki/xz_ate_plot_podnoszenie_miekki_komp_wiertarka.png}
% 	}
% 	\caption{Porównanie trajektorii chwytaka w~osiach $X$ i~$Z$}
% 	\label{fig:w_bok_miekki_porow_przedm_bok}
% \end{figure}

% \begin{figure}[H]
% 	\centering
% 	\subfigure[Rzut na wprost]{
% 		\label{fig:w_bok_miekki_porow_zbiorcze_a}
% 		\includegraphics[width=.47\textwidth]{./velma/przerobione_testy/out/w_bok_miekki/common_yz.png}
% 	}
% 	\hfill
% 	% \subfigure[Rzut z~boku]{
% 	% 	\label{fig:w_bok_miekki_porow_zbiorcze_b}
% 	% 	\includegraphics[width=.47\textwidth]{./velma/przerobione_testy/out/w_bok_miekki/common_xz.png}
% 	% }
% 	\caption{Porównanie wszystkich trajektorii.}
% 	\label{fig:w_bok_miekki_porow_zbiorcze}
% \end{figure}


\section{Ruch do góry}

Eksperyment ma przetestować zachowanie algorytmu kompensacji przy ruchu końcówki do góry (rys. \ref{fig:do_gory_a}, \ref{fig:do_gory_rot}).  Ruch jest interesujący przez działanie siły grawitacji właśnie w~osi pionowej $Z$. Ruch różni się od ruchu podnoszenia przedmiotu, ponieważ grawitacja przedmiotu jest już wstępnie skompensowana. Trajektoria ruchu w~rzucie ATE została zaprezentowana na rys. \ref{fig:do_gory_porow_komp}, \ref{fig:do_gory_porow_przedm}.

\begin{figure}[H]
	\centering
	\subfigure[Oś $X$]{
		\label{fig:do_gory_ax}
		\includegraphics[width=.47\textwidth]{./velma/przerobione_testy/out/do_gory/common_ax.png}
	}
	\hfill
	\subfigure[Oś $Y$]{
		\label{fig:do_gory_ay}
		\includegraphics[width=.47\textwidth]{./velma/przerobione_testy/out/do_gory/common_ay.png}
	}
	
	\subfigure[Oś $Z$]{
		\label{fig:do_gory_az}
		\includegraphics[width=.47\textwidth]{./velma/przerobione_testy/out/do_gory/common_az.png}
	}

	\caption{Ruch do góry. Porównanie trajektorii pozycji w~zależności od czasu.}
	\label{fig:do_gory_a}

\end{figure}
\begin{figure}[H]
	\centering
	\subfigure[Brak algorytmu kompensacji]{
		\includegraphics[width=.47\textwidth]{./velma/przerobione_testy/out/do_gory/xz_ate_plot_podnoszenie_miekki_bez_brak.png}
	}
	\hfill
	\subfigure[Załączony algorytm kompensacji]{
		\includegraphics[width=.47\textwidth]{./velma/przerobione_testy/out/do_gory/xz_ate_plot_podnoszenie_miekki_komp_brak.png}
	}
	\caption{Ruch do góry. Porównanie trajektorii chwytaka w~osiach $X$ i~$Z$}
	\label{fig:do_gory_porow_komp}
\end{figure}


\begin{figure}[H]
	\centering
	\subfigure[Kąt obrotu osi $X$]{
		\label{fig:do_gory_rotx}
		\includegraphics[width=.47\textwidth]{./velma/przerobione_testy/out/do_gory/common_rotx.png}
	}
	\hfill
	\subfigure[Kąt obrotu osi $Y$]{
		\label{fig:do_gory_roty}
		\includegraphics[width=.47\textwidth]{./velma/przerobione_testy/out/do_gory/common_roty.png}
	}
	
	\subfigure[Kąt obrotu osi $Z$]{
		\label{fig:do_gory_rotz}
		\includegraphics[width=.47\textwidth]{./velma/przerobione_testy/out/do_gory/common_rotz.png}
	}

	\caption{Ruch do góry. Porównanie trajektorii kątów w~notacji Eulera w~zależności od czasu.}
	\label{fig:do_gory_rot}

\end{figure}



\begin{figure}[H]
	\centering
	\subfigure[Trajektoria z~chwyconą puszką]{
		\includegraphics[width=.47\textwidth]{./velma/przerobione_testy/out/do_gory/xz_ate_plot_podnoszenie_miekki_komp_piwo.png}
	}
	\hfill
	\subfigure[Trajektoria z~chwyconą wiertarką]{
		\includegraphics[width=.47\textwidth]{./velma/przerobione_testy/out/do_gory/xz_ate_plot_podnoszenie_miekki_komp_wiertarka.png}
	}
	\caption{Ruch do góry. Porównanie trajektorii chwytaka w~osiach $X$ i~$Z$.}
	\label{fig:do_gory_porow_przedm}
\end{figure}


% \begin{figure}
% 	\centering
% 	\subfigure[Brak algorytmu kompensacji]{
% 		\includegraphics[width=.47\textwidth]{./velma/przerobione_testy/out/do_gory/xy_ate_plot_podnoszenie_miekki_bez_brak.png}
% 	}
% 	\hfill
% 	\subfigure[Załączony algorytm kompensacji]{
% 		\includegraphics[width=.47\textwidth]{./velma/przerobione_testy/out/do_gory/xy_ate_plot_podnoszenie_miekki_komp_brak.png}
% 	}
% 	\caption{Porównanie trajektorii chwytaka w~osiach $X$ i~$Y$}
% 	\label{fig:do_gory_porow_komp_bok}
% \end{figure}

% \begin{figure}
% 	\centering
% 	\subfigure[Trajektoria z~chwyconą puszką]{
% 		\includegraphics[width=.47\textwidth]{./velma/przerobione_testy/out/do_gory/xy_ate_plot_podnoszenie_miekki_komp_piwo.png}
% 	}
% 	\hfill
% 	\subfigure[Trajektoria z~chwyconą wiertarką]{
% 		\includegraphics[width=.47\textwidth]{./velma/przerobione_testy/out/do_gory/xy_ate_plot_podnoszenie_miekki_komp_wiertarka.png}
% 	}
% 	\caption{Porównanie trajektorii chwytaka w~osiach $X$ i~$Y$}
% 	\label{fig:do_gory_porow_przedm_bok}
% \end{figure}

% \begin{figure}[H]
% 	\centering
% 	\subfigure[Rzut na wprost]{
% 		\label{fig:do_gory_porow_zbiorcze_a}
% 		\includegraphics[width=.47\textwidth]{./velma/przerobione_testy/out/do_gory/common_xz.png}
% 	}
% 	% \hfill
% 	% \subfigure[Rzut z~gory]{
% 	% 	\label{fig:do_gory_porow_zbiorcze_b}
% 	% 	\includegraphics[width=.47\textwidth]{./velma/przerobione_testy/out/do_gory/common_xy.png}
% 	% }
% 	\caption{Porównanie wszystkich trajektorii.}
% 	\label{fig:do_gory_porow_zbiorcze}
% \end{figure}

\section{Ruch do dołu}

Eksperyment ma przetestować zachowanie algorytmu kompensacji przy ruchu końcówki do dołu (rys. \ref{fig:do_dolu_a}, \ref{fig:do_dolu_rot}). Trajektoria ruchu w osi $Z$ pokrywa się z~kierunkiem działania siły grawitacji. Trajektoria ruchu w~rzucie ATE została zaprezentowana na rys. \ref{fig:do_dolu_porow_komp}, \ref{fig:do_dolu_porow_przedm}.
% Trajektoria widoczna z~boku (w osiach $X$ oraz $Z$) została zaprezentowana na rys. \ref{fig:do_dolu_porow_komp_bok}, \ref{fig:do_dolu_porow_przedm_bok} i~\ref{fig:do_dolu_porow_zbiorcze_b}.

\begin{figure}[H]
	\centering
	\subfigure[Oś $X$]{
		\label{fig:do_dolu_ax}
		\includegraphics[width=.47\textwidth]{./velma/przerobione_testy/out/do_dolu/common_ax.png}
	}
	\hfill
	\subfigure[Oś $Y$]{
		\label{fig:do_dolu_ay}
		\includegraphics[width=.47\textwidth]{./velma/przerobione_testy/out/do_dolu/common_ay.png}
	}
	
	\subfigure[Oś $Z$]{
		\label{fig:do_dolu_az}
		\includegraphics[width=.47\textwidth]{./velma/przerobione_testy/out/do_dolu/common_az.png}
	}

	\caption{Ruch do dołu. Porównanie trajektorii pozycji w~zależności od czasu.}
	\label{fig:do_dolu_a}

\end{figure}

\begin{figure}[H]
	\centering
	\subfigure[Brak algorytmu kompensacji]{
		\includegraphics[width=.47\textwidth]{./velma/przerobione_testy/out/do_dolu/xz_ate_plot_podnoszenie_miekki_bez_brak.png}
	}
	\hfill
	\subfigure[Załączony algorytm kompensacji]{
		\includegraphics[width=.47\textwidth]{./velma/przerobione_testy/out/do_dolu/xz_ate_plot_podnoszenie_miekki_komp_brak.png}
	}
	\caption{Ruch do dołu. Porównanie trajektorii chwytaka w~osiach $X$ i~$Z$}
	\label{fig:do_dolu_porow_komp}
\end{figure}

\begin{figure}[H]
	\centering
	\subfigure[Kąt obrotu osi $X$]{
		\label{fig:do_dolu_rotx}
		\includegraphics[width=.47\textwidth]{./velma/przerobione_testy/out/do_dolu/common_rotx.png}
	}
	\hfill
	\subfigure[Kąt obrotu osi $Y$]{
		\label{fig:do_dolu_roty}
		\includegraphics[width=.47\textwidth]{./velma/przerobione_testy/out/do_dolu/common_roty.png}
	}
	
	\subfigure[Kąt obrotu osi $Z$]{
		\label{fig:do_dolu_rotz}
		\includegraphics[width=.47\textwidth]{./velma/przerobione_testy/out/do_dolu/common_rotz.png}
	}

	\caption{Ruch do dołu. Porównanie trajektorii katów w~notacji Eulera w~zależności od czasu.}
	\label{fig:do_dolu_rot}

\end{figure}



\begin{figure}[H]
	\centering
 	\subfigure[Trajektoria z~chwyconą puszką]{
 		\includegraphics[width=.47\textwidth]{./velma/przerobione_testy/out/do_dolu/xz_ate_plot_podnoszenie_miekki_komp_piwo.png}
	}
 	\hfill
 	\subfigure[Trajektoria z~chwyconą wiertarką]{
 		\includegraphics[width=.47\textwidth]{./velma/przerobione_testy/out/do_dolu/xz_ate_plot_podnoszenie_miekki_komp_wiertarka.png}
 	}
 	\caption{Porównanie trajektorii chwytaka w~osiach $X$ i~$Z$}
 	\label{fig:do_dolu_porow_przedm}
 \end{figure}


% \begin{figure}
% 	\centering
% 	\subfigure[Brak algorytmu kompensacji]{
% 		\includegraphics[width=.47\textwidth]{./velma/przerobione_testy/out/do_dolu/xy_ate_plot_podnoszenie_miekki_bez_brak.png}
% 	}
% 	\hfill
% 	\subfigure[Załączony algorytm kompensacji]{
% 		\includegraphics[width=.47\textwidth]{./velma/przerobione_testy/out/do_dolu/xy_ate_plot_podnoszenie_miekki_komp_brak.png}
% 	}
% 	\caption{Porównanie trajektorii chwytaka w~osiach $X$ i~$Y$}
% 	\label{fig:do_dolu_porow_komp_bok}
% \end{figure}

% \begin{figure}
% 	\centering
% 	\subfigure[Trajektoria z~chwyconą puszką]{
% 		\includegraphics[width=.47\textwidth]{./velma/przerobione_testy/out/do_dolu/xy_ate_plot_podnoszenie_miekki_komp_piwo.png}
% 	}
% 	\hfill
% 	\subfigure[Trajektoria z~chwyconą wiertarką]{
% 		\includegraphics[width=.47\textwidth]{./velma/przerobione_testy/out/do_dolu/xy_ate_plot_podnoszenie_miekki_komp_wiertarka.png}
% 	}
% 	\caption{Porównanie trajektorii chwytaka w~osiach $X$ i~$Y$}
% 	\label{fig:do_dolu_porow_przedm_bok}
% \end{figure}

% \begin{figure}[H]
% 	\centering
% 	\subfigure[Rzut na wprost]{
% 		\label{fig:do_dolu_porow_zbiorcze_a}
% 		\includegraphics[width=.47\textwidth]{./velma/przerobione_testy/out/do_dolu/common_xz.png}
% 	}
% 	\hfill
% 	% \subfigure[Rzut z~gory]{
% 	% 	\label{fig:do_dolu_porow_zbiorcze_b}
% 	% 	\includegraphics[width=.47\textwidth]{./velma/przerobione_testy/out/do_dolu/common_xy.png}
% 	% }
% 	\caption{Porównanie wszystkich trajektorii.}
% 	\label{fig:do_dolu_porow_zbiorcze}
% \end{figure}

\section{Ruch do przodu}

Eksperyment ma przetestować zachowanie algorytmu kompensacji przy ruchu końcówki w~kierunku od robota (rys. \ref{fig:do_przodu_a}, \ref{fig:do_przodu_rot}). Trajektoria ruchu w~rzucie ATE została zaprezentowana na rys. \ref{fig:do_przodu_porow_komp}, \ref{fig:do_przodu_porow_przedm}.
 % i~\ref{fig:do_przodu_porow_zbiorcze_a}. Trajektoria widoczna z~boku (w osiach $X$ oraz $Z$) została zaprezentowana na rys. \ref{fig:do_przodu_porow_komp_bok}, \ref{fig:do_przodu_porow_przedm_bok} i~\ref{fig:do_przodu_porow_zbiorcze_b}.

\begin{figure}[H]
	\centering
	\subfigure[Oś $X$]{
		\label{fig:do_przodu_ax}
		\includegraphics[width=.47\textwidth]{./velma/przerobione_testy/out/do_przodu/common_ax.png}
	}
	\hfill
	\subfigure[Oś $Y$]{
		\label{fig:do_przodu_ay}
		\includegraphics[width=.47\textwidth]{./velma/przerobione_testy/out/do_przodu/common_ay.png}
	}
	
	\subfigure[Oś $Z$]{
		\label{fig:do_przodu_az}
		\includegraphics[width=.47\textwidth]{./velma/przerobione_testy/out/do_przodu/common_az.png}
	}

	\caption{Ruch do przodu. Porównanie trajektorii pozycji w~zależności od czasu.}
	\label{fig:do_przodu_a}

\end{figure}
\begin{figure}[H]
	\centering
	\subfigure[Brak algorytmu kompensacji]{
		\includegraphics[width=.47\textwidth]{./velma/przerobione_testy/out/do_przodu/xz_ate_plot_podnoszenie_miekki_bez_brak.png}
	}
	\hfill
	\subfigure[Załączony algorytm kompensacji]{
		\includegraphics[width=.47\textwidth]{./velma/przerobione_testy/out/do_przodu/xz_ate_plot_podnoszenie_miekki_komp_brak.png}
	}
	\caption{Ruch do przodu. Porównanie trajektorii chwytaka w~osiach $X$ i~$Z$}
	\label{fig:do_przodu_porow_komp}
\end{figure}


\begin{figure}[H]
	\centering
	\subfigure[Kąt obrotu osi $X$]{
		\label{fig:do_przodu_rotx}
		\includegraphics[width=.47\textwidth]{./velma/przerobione_testy/out/do_przodu/common_rotx.png}
	}
	\hfill
	\subfigure[Kąt obrotu osi $Y$]{
		\label{fig:do_przodu_roty}
		\includegraphics[width=.47\textwidth]{./velma/przerobione_testy/out/do_przodu/common_roty.png}
	}
	
	\subfigure[Kąt obrotu osi $Z$]{
		\label{fig:do_przodu_rotz}
		\includegraphics[width=.47\textwidth]{./velma/przerobione_testy/out/do_przodu/common_rotz.png}
	}

	\caption{Ruch do przodu. Porównanie trajektorii katów w~notacji Eulera w~zależności od czasu.}
	\label{fig:do_przodu_rot}

\end{figure}



\begin{figure}[H]
	\centering
	\subfigure[Trajektoria z~chwyconą puszką]{
		\includegraphics[width=.47\textwidth]{./velma/przerobione_testy/out/do_przodu/xz_ate_plot_podnoszenie_miekki_komp_piwo.png}
	}
	\hfill
	\subfigure[Trajektoria z~chwyconą wiertarką]{
		\includegraphics[width=.47\textwidth]{./velma/przerobione_testy/out/do_przodu/xz_ate_plot_podnoszenie_miekki_komp_wiertarka.png}
	}
	\caption{Ruch do przodu. Porównanie trajektorii chwytaka w~osiach $X$ i~$Z$}
	\label{fig:do_przodu_porow_przedm}
\end{figure}


% \begin{figure}[H]
% 	\centering
% 	\subfigure[Brak algorytmu kompensacji]{
% 		\includegraphics[width=.47\textwidth]{./velma/przerobione_testy/out/do_przodu/xy_ate_plot_podnoszenie_miekki_bez_brak.png}
% 	}
% 	\hfill
% 	\subfigure[Załączony algorytm kompensacji]{
% 		\includegraphics[width=.47\textwidth]{./velma/przerobione_testy/out/do_przodu/xy_ate_plot_podnoszenie_miekki_komp_brak.png}
% 	}
% 	\caption{Porównanie trajektorii chwytaka w~osiach $X$ i~$Y$}
% 	\label{fig:do_przodu_porow_komp_bok}
% \end{figure}

% \begin{figure}[H]
% 	\centering
% 	\subfigure[Trajektoria z~chwyconą puszką]{
% 		\includegraphics[width=.47\textwidth]{./velma/przerobione_testy/out/do_przodu/xy_ate_plot_podnoszenie_miekki_komp_piwo.png}
% 	}
% 	\hfill
% 	\subfigure[Trajektoria z~chwyconą wiertarką]{
% 		\includegraphics[width=.47\textwidth]{./velma/przerobione_testy/out/do_przodu/xy_ate_plot_podnoszenie_miekki_komp_wiertarka.png}
% 	}
% 	\caption{Porównanie trajektorii chwytaka w~osiach $X$ i~$Y$}
% 	\label{fig:do_przodu_porow_przedm_bok}
% \end{figure}

% \begin{figure}[H]
% 	\centering
% 	\subfigure[Rzut na wprost]{
% 		\label{fig:do_przodu_porow_zbiorcze_a}
% 		\includegraphics[width=.47\textwidth]{./velma/przerobione_testy/out/do_przodu/common_xz.png}
% 	}
% 	\hfill
% 	\subfigure[Rzut z~gory]{
% 		\label{fig:do_przodu_porow_zbiorcze_b}
% 		\includegraphics[width=.47\textwidth]{./velma/przerobione_testy/out/do_przodu/common_xy.png}
% 	}
% 	\caption{Porównanie wszystkich trajektorii bez zaznaczonego błędu}
% 	\label{fig:do_przodu_porow_zbiorcze}
% \end{figure}


\section{Obrót końcówki}

Eksperyment ma przetestować zachowanie algorytmu kompensacji przy obrocie końcówki. Obserwacje polegają na zmianie położeń kątowych końcówki bez zmiany pozycji. Końcówka ma obrócić się o~zadany kat we wszystkich osiach (rys.\ref{fig:obrt_a}, \ref{fig:obrt_rot}) Trajektoria ruchu (w osiach $X$ oraz $Y$) została zaprezentowana na rys. \ref{fig:obrt_porow_komp} i~\ref{fig:obrt_porow_przedm}. Trajektoria widoczna z~boku (w osiach $X$ oraz $Z$) została zaprezentowana na rys. \ref{fig:obrt_porow_komp_bok}, \ref{fig:obrt_porow_przedm_bok}.

\begin{figure}[H]
	\centering
	\subfigure[Oś $X$]{
		\label{fig:obrt_ax}
		\includegraphics[width=.47\textwidth]{./velma/przerobione_testy/out/obrt/common_ax.png}
	}
	\hfill
	\subfigure[Oś $Y$]{
		\label{fig:obrt_ay}
		\includegraphics[width=.47\textwidth]{./velma/przerobione_testy/out/obrt/common_ay.png}
	}
	
	
	\subfigure[Oś $Z$]{
		\label{fig:obrt_az}
		\includegraphics[width=.47\textwidth]{./velma/przerobione_testy/out/obrt/common_az.png}
	}

	\caption{Porównanie trajektorii pozycji w~zależności od czasu.}
	\label{fig:obrt_a}

\end{figure}

\begin{figure}[H]
	\centering
	\subfigure[Brak algorytmu kompensacji]{
		\includegraphics[width=.47\textwidth]{./velma/przerobione_testy/out/obrt/xz_ate_plot_podnoszenie_miekki_bez_brak.png}
	}
	\hfill
	\subfigure[Załączony algorytm kompensacji]{
		\includegraphics[width=.47\textwidth]{./velma/przerobione_testy/out/obrt/xz_ate_plot_podnoszenie_miekki_komp_brak.png}
	}
	\caption{Porównanie trajektorii chwytaka w~osiach $X$ i~$Z$}
	\label{fig:obrt_porow_komp}
\end{figure}

\begin{figure}[H]
	\centering
	\subfigure[Kąt obrotu osi $X$]{
		\label{fig:obrt_rotx}
		\includegraphics[width=.47\textwidth]{./velma/przerobione_testy/out/obrt/common_rotx.png}
	}
	\hfill
	\subfigure[Kąt obrotu osi $Y$]{
		\label{fig:obrt_roty}
		\includegraphics[width=.47\textwidth]{./velma/przerobione_testy/out/obrt/common_roty.png}
	}
	

	\subfigure[Kąt obrotu osi $Z$]{
		\label{fig:obrt_rotz}
		\includegraphics[width=.47\textwidth]{./velma/przerobione_testy/out/obrt/common_rotz.png}
	}

	\caption{Porównanie trajektorii kątów w~notacji Eulera w~zależności od czasu.}
	\label{fig:obrt_rot}

\end{figure}




\begin{figure}[H]
	\centering
	\subfigure[Trajektoria z~chwyconą puszką]{
		\includegraphics[width=.47\textwidth]{./velma/przerobione_testy/out/obrt/xz_ate_plot_podnoszenie_miekki_komp_piwo.png}
	}
	\hfill
	\subfigure[Trajektoria z~chwyconą wiertarką]{
		\includegraphics[width=.47\textwidth]{./velma/przerobione_testy/out/obrt/xz_ate_plot_podnoszenie_miekki_komp_wiertarka.png}
	}
	\caption{Porównanie trajektorii chwytaka w~osiach $X$ i~$Z$}
	\label{fig:obrt_porow_przedm}
\end{figure}


\begin{figure}[H]
	\centering
	\subfigure[Brak algorytmu kompensacji]{
		\includegraphics[width=.47\textwidth]{./velma/przerobione_testy/out/obrt/xy_ate_plot_podnoszenie_miekki_bez_brak.png}
	}
	\hfill
	\subfigure[Załączony algorytm kompensacji]{
		\includegraphics[width=.47\textwidth]{./velma/przerobione_testy/out/obrt/xy_ate_plot_podnoszenie_miekki_komp_brak.png}
	}
	\caption{Porównanie trajektorii chwytaka w~osiach $X$ i~$Y$}
	\label{fig:obrt_porow_komp_bok}
\end{figure}

\begin{figure}[H]
	\centering
	\subfigure[Trajektoria z~chwyconą puszką]{
		\includegraphics[width=.47\textwidth]{./velma/przerobione_testy/out/obrt/xy_ate_plot_podnoszenie_miekki_komp_piwo.png}
	}
	\hfill
	\subfigure[Trajektoria z~chwyconą wiertarką]{
		\includegraphics[width=.47\textwidth]{./velma/przerobione_testy/out/obrt/xy_ate_plot_podnoszenie_miekki_komp_wiertarka.png}
	}
	\caption{Porównanie trajektorii chwytaka w~osiach $X$ i~$Y$}
	\label{fig:obrt_porow_przedm_bok}
\end{figure}

%\begin{figure}[H]
%	\centering
%	\subfigure[Rzut na wprost]{
%		\label{fig:obrt_porow_zbiorcze_a}
%		\includegraphics[width=.47\textwidth]{./velma/przerobione_testy/out/obrt/common_xz.png}
%	}
%	\hfill
%	\subfigure[Rzut z~gory]{
%		\label{fig:obrt_porow_zbiorcze_b}
%		\includegraphics[width=.47\textwidth]{./velma/przerobione_testy/out/obrt/common_xy.png}
%	}
%	\caption{Porównanie wszystkich trajektorii bez zaznaczonego błędu}
%	\label{fig:obrt_porow_zbiorcze}
%\end{figure}

%\begin{figure}[H]
%	\centering
%	\subfigure[Trajektoria z~chwyconą puszką]{
%		\includegraphics[width=.47\textwidth]{./velma/przerobione_testy/out/obrt/xy_ate_plot_podnoszenie_miekki_komp_piwo.png}
%	}
%	\hfill
%	\subfigure[Trajektoria z~chwyconą wiertarką]{
%		\includegraphics[width=.47\textwidth]{./velma/przerobione_testy/out/obrt/xy_ate_plot_podnoszenie_miekki_komp_wiertarka.png}
%	}
%	\caption{Porównanie trajektorii chwytaka w~osiach $X$ i~$Y$}
%	\label{fig:obrt_porow_przedm_bok}
%\end{figure}


\section{Wnioski}

Metryka APE jest dobrym narzędziem do oceny jakości trajektorii końcówki. Ruchy tego samego typu mają zbliżony do siebie współczynnik RMSE (tab. \ref{tab:}). 

Wyniki trajektorii przeprowadzonych bez modyfikacji prawa sterowania zostały porównane do trajektorii z zaimplementowanym prawem sterowania. Celem algorytmu kompensacji jest doprowadzenie wszystkich trajektorii do możliwie podobnej postaci, jak trajektoria bez zmian w prawie sterowania. 

Zmodyfikowane prawo sterowania pozwala na skuteczną kompensacje siły grawitacji chwyconego przedmiotu. Modyfikacja pozwala na kompensację nie tylko sił grawitacji, ale także na ograniczenie innych niedoskonałości modelu stosowanego w~prawie sterowania. Osiągane błędy średnio-kwadratowe wykonywanych trajektorii są rzędu pojedynczych centymetrów i~maksymalnie około dwa razy większe niż w~przypadku wzorcowych trajektorii bez algorytmu kompensacji i~bez chwyconego przedmiotu. (tab. \ref{tab:}). 

W trakcie wszystkich eksperymentów widać, że prawo sterowania osiąga mniejsze błędy dla prostszego przedmiotu, jakim jest puszka z~jednorodnym rozkładem mas. Algorytm kompensacji wprowadza oscylacje zarówno położenia jak i~rotacji. Zjawisko jest najbardziej widoczne gdy prędkość w~końcówce jest wytracana. Przez właściwości zaimplementowanego prawa sterowania podnoszenie przedmiotów trwa długo. 

Empiryczne eksperymenty w~trakcie pracy symulatora wykazały, że wprowadzone modyfikacje nie miały dużego wpływu na uginanie się robota w~trakcie kolizji z~przedmiotami, chyba że kolizja trwa bardzo długo. Wytłumaczeniem tego pozytywnego zjawiska może być fakt powolnej kompensacji po chwyceniu przedmiotu. Robot potrzebuje ok 15 s, aby dodać do końcówki chwytaka siłę o~wartości odpowiadającej sile ciężkości jednego kilograma. 

{\small
\begin{table}[H]

	\begin{tabular}{||c | c c c c ||}

		\hline

		Opis ruchu --  Błąd [m]  &  Bez alg. komp. & Bez przedmiotu & Z~puszką  & Z~wiertarką  \\ [0.5ex]

		\hline\hline

		Podnoszenie & 0.032490 & 0.053853 & 0.090244 & 0.104056 \\
		Ruch ósemkowy &0.102113  & 0.092006 & 0.089532 & 0.095919 \\
		Ruch w~bok & 0.030680 & 0.040400 & 0.041024 &  0.047411\\
		Ruch do góry & 0.035023 & 0.051653 & 0.050177& 0.048357 \\
		Ruch do dołu & 0.038727 & 0.055311 & 0.055371 & 0.057347 \\
		Ruch do przodu & 0.035068 & 0.060216 & 0.058322 & 0.056084 \\
		Obrót końcówki & 0.005984 & 0.007722 & 0.038998 & 0.140615\\
		\hline

	\end{tabular}

	\caption{Porównanie błędu średnio-kwadratowego RMSE w~metryce APE dla poszczególnych ruchów.}

	\label{tab:}

\end{table}
}