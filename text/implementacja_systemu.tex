% encoding: utf8
% !TEX encoding = utf8
% !TeX spellcheck = pl_PL

\chapter{Implementacja systemu \label{chap:specyfikacja_systemu}}
W celu rozwiązania problemu opisanego w~pracy dokonano szeregu modyfikacji w~systemie oprogramowania robota Velma. Zmodyfikowano prawo sterowania impedancyjnego, tak aby dodać człon całkujący. Dodano możliwość konfiguracji nowych parametrów prawa sterowania związanych z~członem całkującym. Stworzono środowisko pozwalające na śledzenie oraz wizualizację wymaganych trajektorii chwytaka.

\section{Prawo sterowania}
Modyfikacja prawa sterowania została wykonana w~komponencie \textit{cart\_imp} znajdującym się w podsystemie \textit{velma\_core\_cs}. Ponieważ mamy do czynienia z~systemem z~dyskretnym czasem, nie jest możliwe bezpośrednie zaimplementowanie wzoru \ref{eq:prawo_ster}. Człony związane z~impedancyjnym prawem sterowania były już zaimplementowane. Do implementacji członu całkującego skorzystano z~dyskretnej wersji regulatora PID \cite{wiki:PID_controller}. Dla przypadku jednowymiarowego w~chwili $k$ prawo sterowania jest postaci:
\begin{equation}
\label{eq:dysk}
u(k) = u(k-1) + P[(1 + \frac{\Delta}{T_i} + \frac{T_d}{\Delta})e(k) + (-1-\frac{2T_d}{\Delta})e(k-1) + \frac{T_d}{\Delta}e(k-2)]
\end{equation}
przy założeniach:
\begin{equation}
T_i = P/I
\end{equation}
\begin{equation}
T_d = D/P
\end{equation}

gdzie:
\begin{itemize}
	\item $u(k)$ to sterowanie w~chwili $k$.
	\item $e(k)$ to uchyb w~chwili $k$.
	\item $\Delta$ to czas jednego kroku symulacji, dla systemu robota Velma ma wartość $\frac{1}{500}$~s.
	\item $P$ to parametr członu proporcjonalnego.
	\item $I$ to parametr członu całkującego.
	\item $D$ to parametr członu różniczkującego.
\end{itemize}
Dla naszych potrzeb istotny jest tylko człon całkujący więc równanie \ref{eq:dysk} upraszcza się do postaci:
\begin{equation}
\label{eq:dysk_cal}
	u(k) = u(k-1) + e(k)+\frac{\Delta e(k)}{T_i} - e(k-1) -\frac{2T_de(k-1)}{\Delta} 
\end{equation}

Następnie wzór \ref{eq:dysk_cal} jest rozszerzany do postaci sześciowymiarowego wektora poprzez zastosowanie w~każdym wierszu tego wektora. Taki wektor można łatwo dodać do istniejącego prawa sterowania w~przestrzeni operacyjnej robota.

\section{Parametry członu całkującego}
W celu umożliwienia konfiguracji parametrów członu całkującego został zmodyfikowany interfejs \textit{VelmaInterface}. Do ruchu ramieniem w~przestrzeni operacyjnej służy funkcja \textit{moveCartImpRight()}. Jej nowy nagłówek został rozszerzony o~parametr \textit{gcomp}, który ma być listą opisującą przekątną macierzy członu całkowania. Opisane parametry są przekazywane przez system akcji ROSa do agenta \textit{velma\_task\_cs\_ros\_interface}. Odpowiedzialny za to jest komponent \textit{CartImpActionRight}, który został odpowiednio zmodyfikowany w~ramach pracy. Komponent przyjmuje zmodyfikowaną w ramach pracy wiadomość \textit{RightImpAction}, która zawiera odpowiednie parametry. 
 
\section{Rejestracja trajektorii}
Ocena jakości algorytmu sterowania następuje poprzez porównanie odpowiednich trajektorii. W~celu wydobycia trajektorii określonych przez wektory siły uogólnionej zmodyfikowano komponent \textit{TfPublisher} z~podsystemu \textit{velma\_task\_cs\_ros\_interface} w~taki sposób, żeby udostępnił pozycję zadaną, generowaną przez interpolator trajektorii na odpowiednim temacie ROS o~nazwie \textit{right\_arm\_cmd}. W~podobny sposób udostępniona jest pozycja osiągnięta w~danym momencie przez robota w~temacie \textit{right\_arm\_tool}. Za pomocą pomocniczego programu \textit{writer.py} pisanego w~Pythonie pozycje są odbierane w~trakcie działania systemu i~zapisywane wraz ze znacznikiem czasu.


\section{Generowanie wykresów}
Zapisane pliki z~danymi są przetwarzane przez napisany program \textit{plotter.py} zaimplementowany w~Pythonie. Jego zadaniem jest generowanie obrazów pozwalających na porównanie zapisanych w~plikach trajektorii oraz wyliczeń związanych z~oceną jakości trajektorii na podstawie metryki APE. Program tworzy przebiegi w~zależności od czasu dla konkretnych osi oraz kątów wielu trajektorii. Ma też możliwość generowania dwuwymiarowych rzutów metryki APE przez wywołanie zewnętrznego skryptu \cite{bib:evaluate_ape}. Program \textit{plotter.py} może porównywać dowolną liczbę trajektorii. Pliki, które podlegają obliczeniom są podawane w~argumentach programu.


\section{Opis świata testowego}
Stworzono nowe pliki \textit{can\_world.world} oraz \textit{drill\_world.world} które są opisem świata używanym przez symulator świata Gazebo. Pliki są strukturami typu XML, które zawierają opis i~umiejscowienie w~świecie konkretnych przedmiotów. Obydwa pliki opisują świat z~robotem Velma wraz ze znajdującym się obok niego stole. Na stole położone są przedmioty testowe, którymi są puszka oraz wiertarka.

\section{Algorytm eksperymentu}
Do testowania modyfikacji systemu stworzono program \textit{test\_script.py}, który wykorzystuje interfejs \textit{VelmaInterface} i~określa ruchy, które ma wykonać robot w~trakcie testów. Program wykonuje poniższe czynności:
\begin{enumerate}
	\item Inicjalizację obiektów potrzebnych do komunikacji z~ROSem.
	\item Ułożenie robota w~odgórnie zdefiniowanej pozycji startowej.
	\item Ruch ramienia do pozycji, w~której może chwycić przedmiot.
	\item Chwycenie przedmiotu poprzez zaciśnięcie chwytaka.
	\item Ruch ramienia do pozycji rozpoczęcia następnego testu.
	\item Ruch ósemkowy wykonany chwytakiem.
	\item Ruch do góry i~z powrotem wykonany chwytakiem.
	\item Ruch w~bok i~z powrotem wykonany chwytakiem.
	\item Ruch do przodu i~z powrotem wykonany chwytakiem.
	\item Ruch obrotu wykonany chwytakiem. 
\end{enumerate}
